\documentclass[scheme=plain]{ctexart}
\usepackage{hyperref}

\title{陶云峰的简历}

\pagestyle{empty}
\ctexset {
    today = small,
}

\begin{document}
\maketitle

\section{基本信息}

陶大,字云峰,松江府上海县人士。
码代码三十有二年矣。
而立前得高人点化,兼修数理逻辑并经济学,擅模型论与拍卖论。
后恰云计算军兴,从阿里云转战调度、一致性、数据库诸路。
平生嗜读书与吃肉,然则年迈则精神日蹙,娶妻则收敛饕餮之胃,终无廉颇之勇矣。

\begin{itemize}
    \item email: tyf00@aliyun.com
    \item phone: 13641642749
    \item 居住地:上海
    \item 出生年份:1980年
    \item github: \url{https://github.com/TimeExceed}
    \item 知乎: \url{https://www.zhihu.com/people/tao-yun-feng}
\end{itemize}

\section{Highlights}

在依图基础设施部主持了CodeCraft和Gringotts两个项目的设计和开发。

\subsection*{CodeCraft}

CodeCraft是一个内部使用的数据安全产品。
整件事情的难点不在这个那个技术,而在细致周到。
我们结合了依图数据使用的现状,确定了以下设计:
\begin{itemize}
    \item 一个CodeCraft实例是一个独立的网络域,其内部互通而对外除非预先设计过的通路,一律切断。
    \item 用户通过远程桌面进入CodeCraft实例工作。并且有方便的数据进入通路。
        \begin{itemize}
            \item 我们识别出用户的日常工作经常会去下载一些什么,比如\textsl{apt install}或者\textsl{git clone}。
                为此我们专门设计了HTTP的通路。
        \end{itemize}
    \item 有专门设计的数据出来的通路,满足审批、审计等需要。
\end{itemize}

\subsection*{Gringotts}

Gringotts是一个数据集的实现。

依图训练用数据大量以文件形式存放在CephFS上。
这带来了许多问题:
\begin{itemize}
    \item 大量的小文件导致存储放大严重。
    \item 大量的小文件导致对CephFS元数据的压力。
    \item 文件本身缺少合适的管理。
        比如我现在要训练一个识别男人的模型,那么如何从数billion的文件里把包含男人的照片摘选出来呢?
        文件系统本身对这块并没有针对性的设计。
    \item 具体到CephFS,缺少对文件内容的版本管理。
        故而算法工程师必须小心谨慎地使用。
\end{itemize}
然而小文件使用便利,生态周全。
比如GitHub上有大量算法直接读取jpeg,算法工程师复制下来直接就可以跑。

为了解决以上问题,我们设计了
\begin{itemize}
    \item 每个数据集都是Btrfs over RBD。
        由此既保持了文件的界面,又达到了极小的存储放大。
    \item 在使用中,用户对数据的写入会保存在本地的OverlayFS内。
        基于此,我们实现了完整的数据集事务和分支。
    \item 最后,我们将每个数据集内的数据的元数据和标注都导入搜索引擎。
        用户在命令行上发出搜索的指令,其结果会自动出现在指定的目录下(FUSE挂载)。
\end{itemize}

\section{技能点}

\subsection{软件技术}

擅长技术写作和技术讲座。
平生未有一败。
曾被阿里云评为最佳讲师。

\begin{enumerate}
    \item C++
        \begin{itemize}
            \item 为云栖社区准备的C++讲座(与人合讲): \url{https://developer.aliyun.com/article/581180}
            \item 阿里云表格存储的C++ SDK: \url{https://github.com/TimeExceed/aliyun-tablestore-cpp-sdk}
        \end{itemize}
    \item Rust
        \begin{itemize}
            \item 在依图硬件部落地。
            \item 一个面向架构师的介绍:\url{https://github.com/TimeExceed/cert_rust}
        \end{itemize}
    \item Python
        \begin{itemize}
            \item 从2.3版本开始入手。
            \item 一个自用的做图工具:\url{https://github.com/TimeExceed/fathom}
        \end{itemize}
    \item 其他编程语言可以看\href{https://www.zhihu.com/question/403828823/answer/1310069487}{这篇回答}。
    \item 系统
        \begin{itemize}
            \item 在阿里云期间从事过分布式调度(伏羲)、一致性(女娲)、存储和数据库(表格存储)的研发。
                \begin{itemize}
                    \item 一份面向新人的存储技术介绍:\url{https://github.com/TimeExceed/modern_db_part0}
                \end{itemize}
            \item 在依图主管三年的嵌入式(摄像头、边缘盒子)软件研发。
            \item 在依图主管半年PaaS(K8S)研发。
        \end{itemize}
    \item 软件工程
        \begin{itemize}
            \item 整体观点:\url{https://github.com/TimeExceed/design-principles}
            \item 项目管理的Kanban方法: \url{https://github.com/TimeExceed/kanban}
            \item 测试:
                \begin{itemize}
                    \item 测试理论:\url{https://github.com/TimeExceed/test-theory-intro}
                    \item TDD: \url{https://github.com/TimeExceed/tdd_slides}
                \end{itemize}
        \end{itemize}
\end{enumerate}

\subsection{管理}

擅长把草台班子带成正规军。
\begin{itemize}
    \item 2012-2013年(阿里云)带女娲(分布式一致性)团队。
        其中一人现已升为P8。
    \item 2018-2020年接手依图硬件部的软件研发团队。
        接手之初,团队士气很高,但做事没有章法、没有缓急,加班很多而效率不高。
        用三个月时间摸清了团队能力的短板,针对性地制订了技术战略和招聘战略。
        在2019年团队扩充到近30人,有力地完成了三条战略产品线的研发任务。
    \item 2020年接手PaaS和运维平台两个团队。
        我敏锐地发现前任对两个团队的边界的划分不当,推动两个团队合并,同时免去运维平台小组长的管理职务。
        完成以上两个动作没有引起团队士气的任何问题。
\end{itemize}


\section{履历}

\subsection*{2020.5-2021.5 依图/基础设施}

\begin{itemize}
    \item 三人领导小组成员。
        分管软件研发。
        推动部门向内部服务商转型,要点是谈钱。
        \begin{itemize}
            \item Philosophy, principle, plan三人各取一端。
                我是那个philosopher。
        \end{itemize}
    \item 期间领导了依图自用的PaaS(K8S)、训练用存储系统的研发。
\end{itemize}

\subsection*{2018.6-2020.4 依图/硬件产品}

\begin{itemize}
    \item 主持多款硬件产品(门禁、摄像头、边缘盒子)的软件研发。
        \begin{itemize}
            \item 力主转换技术栈到Rust。
                达到了no crash和low overhead两个目标。
        \end{itemize}
    \item 设计并写作依图统一API(OneAPI)
\end{itemize}

\subsection*{2013.12-2018.5 阿里云/表格存储}

\begin{itemize}
    \item 期间将Software Transactional Memory和Immutable Data Structures引入表格存储。
    \item 作为架构师参与女娲向全球一致性(内部)服务转型。
\end{itemize}

\subsection*{2013.5-2013.11 费沙科技}

\begin{itemize}
    \item 作为技术负责人参与别人创业。后创业失败。
    \item 期间选择Clojure和ClojureScript作为后端和前端的核心技术栈。
    \item 在Docker的早期就开始关注,并用于CI。
\end{itemize}

\subsection*{2011.9-2013.4 阿里云/飞天/伏羲-女娲}

\begin{itemize}
    \item P7入职,半年后升P8。
        至今我的晋升PPT还作为升职范本在阿里内部流传。
    \item 作为架构师参与伏羲(分布式调度)重构,为后面支持5K集群做好铺垫。
    \item 主管女娲(一致性)。
        主持女娲迁移Zookeeper。
\end{itemize}

\subsection*{2007-2011.8 纳拓科技}

\begin{itemize}
    \item 一家EDA公司,后被Synopsis收购。我以实习生身份加入。日常使用C++开发。
    \item 作了大量的talks。
\end{itemize}

\subsection*{1998-2011 上海交通大学}

\begin{itemize}
    \item 2000年,作为参赛队员参加ACM/ICPC World Final,获得第7名。实现亚洲队伍前十的突破。
    \item 2002年,交大ACM队获得总决赛冠军。我是常务教练。
    \item 博士期间师从沈恩绍教授,学习数理逻辑和自动机理论。
        中后期转到博弈论、拍卖论和机制设计方向,并以在线广告相关研究获得学位。
\end{itemize}

\appendix
\clearpage
\section{专利}

\subsection*{依图时期}

\begin{enumerate}
    \item CN111770491A/CN202010511943.6 一种数据链路建立方法及装置
    \item CN111581410A/CN202010475291.5 图像检索方法及其装置、介质和系统
    \item CN111615070A/CN202010467753.9 一种业务处理方法及装置
    \item CN111563926A/CN202010446016.0 测量图像中物体物理尺寸的方法、电子设备、介质及系统
    \item CN111586066A/CN202010397155.9 一种多媒体数据加密处理的方法及装置
    \item CN111414497A/CN202010348078.8 一种分布式数据融合处理的方法及装置
    \item CN111510658A/CN202010326765.X 图片数据处理方法、装置、电子设备及存储介质
    \item CN111491369A/CN202010317690.9 一种基于安防前端设备的终端定位方法及装置
    \item CN111399886A/CN202010285926.5 用于设备快速升级的方法及系统
    \item CN111447463A/CN202010254256.0 视频流格式转换及播放控制的方法、装置、介质及其系统
    \item CN111444373A/CN202010234745.X 图像检索方法及其装置、介质和系统
    \item CN110662047A/CN201910956997.0 图像存储方法及装置、电子设备以及计算机存储介质
    \item CN110677699A/CN201910956998.5 视频流和/或图片流数据的共享方法、装置及电子设备
    \item CN110582010A/CN201910939027.X 视频/图片加密传输方法、装置、电子设备及存储介质
    \item CN110717428A/CN201910923395.5 一种融合多个特征的身份识别方法、装置、系统、介质及设备
    \item CN110730216A/CN201910904367.9 图片推送方法、装置、电子设备及计算机可读存储介质
\end{enumerate}

\subsection*{阿里时期}

\begin{enumerate}
    \item CN110019514A/CN201711083268.6 数据同步方法、装置以及电子设备
    \item CN107438092A/CN201710140500.9 用于分布式场景中数据处理的方法和设备
    \item CN107438092B/CN201710140500.9 用于分布式场景中数据处理的方法和设备
    \item CN107196988A/CN201710139662.0 一种跨地域数据传输的方法和设备
    \item CN107196988B/CN201710139662.0 一种跨地域数据传输的方法和设备
    \item CN107145396A/CN201710117678.1 分布式锁实现方法和设备
    \item CN107145396B/CN201710117678.1 分布式锁实现方法和设备
    \item CN108345617A/CN201710060057.4 一种数据同步方法、装置以及电子设备
    \item CN108347455A/CN201710053030.2 元数据交互方法及系统
    \item CN108347454A/CN201710053029.X 元数据交互方法及系统
    \item CN108347454B/CN201710053029.X 元数据交互方法及系统
    \item CN107438092B/CN:201710140500:A Data processing method and apparatus applied to distributed scenes
    \item CN107196988B/CN:201710139662:A Cross-area data transmission method and equipment
    \item CN107145396B/CN:201710117678:A Distributed lock realization method and equipment
    \item CN108347455B/CN201710053030.2 元数据交互方法及系统
    \item CN108280080A/CN201710009380.9 一种数据同步方法、装置以及电子设备
    \item CN107919977A/CN201610888948.4 一种基于Paxos协议的分布式一致性系统的在线扩容、在线缩容的方法和装置
    \item CN107872395A/CN201610843856.4 流量控制方法及设备
    \item CN107786527A/CN201610792479.6 实现服务发现的方法及设备
    \item CN107783860A/CN201610792603.9 一种数据传输的恢复点目标监控方法及设备
    \item CN107800733A/CN201610791965.6 分布式系统中会话标识的生成方法及设备
    \item CN107800733B/CN201610791965.6 分布式系统中会话标识的生成方法及设备
    \item CN107800733B/CN:201610791965:A Generating method of session identification in distributed system and equipment XXX
    \item CN107517227A/CN201610424538.4 用于分布式一致性系统的会话实现方法以及装置
    \item CN107517227B/CN201610424538.4 用于分布式一致性系统的会话实现方法以及装置
    \item CN107124324A/CN201610105054.3 一种基于租约的心跳协议方法和设备
    \item CN107124324B/CN201610105054.3 一种基于租约的心跳协议方法和设备
    \item CN107517227B/CN:201610424538:A Session implementation method and device for distributed consistency system
    \item CN107124324B/CN:201610105054:A Heartbeat protocol method and device based on lease
    \item CN106933547A/CN201511017566.6 全局信息获取及处理的方法、装置和更新系统
    \item CN106933548A/CN201511017614.1 全局信息获取、处理及更新、方法、装置和系统
    \item CN106933550A/CN201511017784.X 全局信息获取、处理及更新方法、装置和系统
    \item CN106933547B/CN201511017566.6 全局信息获取及处理的方法、装置和更新系统
    \item CN106933550B/CN201511017784.X 全局信息获取、处理及更新方法、装置和系统
    \item CN106933548B/CN201511017614.1 全局信息获取、处理及更新、方法、装置和系统
    \item CN106933547B/CN:201511017566:A Global information obtaining method and apparatus, global information processing method and apparatus, and global information updating system
    \item CN106933550B/CN:201511017784:A Method and device for obtaining, processing and updating global information and global information update system
    \item CN106933548B/CN:201511017614:A Global information obtaining method, apparatus and system, global information processing method, apparatus and system, and global informatin updating method, apparatus and system
    \item CN106572130A/CN201510648700.6 用于实现分布式锁管理的方法和设备
    \item CN106572051A/CN201510648902.0 分布式系统中分布式锁服务实现方法以及装置
    \item CN106572054A/CN201510650637.X 分布式系统中分布式锁服务实现方法以及装置
    \item CN106572054B/CN201510650637.X 分布式系统中分布式锁服务实现方法以及装置
    \item CN106572130B/CN201510648700.6 用于实现分布式锁管理的方法和设备
    \item CN106572054B/CN:201510650637:A Distributed lock service realization method and device for distributed system
    \item CN106572130B/CN:201510648700:A Method used for realizing distributed lock management and equipment thereof
    \item CN106534227A/CN201510570429.9 用于扩展分布式一致性服务的方法和设备
    \item CN106534227B/CN201510570429.9 用于扩展分布式一致性服务的方法和设备
    \item CN106453444A/CN201510476786.9 缓存数据共享的方法及设备
    \item CN106453444B/CN201510476786.9 缓存数据共享的方法及设备
    \item CN106406925A/CN201510481261.4 用于支持在线升级的设备和方法
    \item CN106534227B/CN:201510570429:A Method and device of expanding distributed consistency service
    \item CN106453444B/CN:201510476786:A Cache data sharing method and equipment
\end{enumerate}

\end{document}
