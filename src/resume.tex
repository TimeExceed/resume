\documentclass[scheme=plain]{ctexart}
\usepackage{hyperref}

\title{陶云峰的简历}

\pagestyle{empty}
\ctexset {
    today = small,
}

\begin{document}
\maketitle

\section{基本信息}

陶大,字云峰,松江府上海县人士。
码代码三十有二年矣。
而立前得高人点化,兼修数理逻辑并经济学,擅模型论与拍卖论。
后恰云计算军兴,从阿里云转战调度、一致性、数据库诸路。
平生嗜读书与吃肉,然则年迈则精神日蹙,娶妻则收敛饕餮之胃,终无廉颇之勇矣。

\begin{itemize}
    \item email: tyf00@aliyun.com
    \item phone: 13641642749
    \item 居住地:上海
    \item 出生年份:1980年
    \item github: \url{https://github.com/TimeExceed}
    \item 知乎: \url{https://www.zhihu.com/people/tao-yun-feng}
\end{itemize}

\section{关于硬件产品云端一体的若干思考}

今天来看,孤立的硬件产品已经很少见了。
云和端的联动成为主流。
那么架构中有三个组件需要思考:
\begin{enumerate}
    \item 端上关注稳定性和性能,兼顾升级的便利性
    \item 云上关注接入设备的数量和多样性
    \item 云和端之间的网络的可达性和性能
\end{enumerate}

\subsection{端上关注稳定性和性能,兼顾升级的便利性}

我在依图的实践大致可以分成技术手段和管理手段。
技术手段包括:
\begin{itemize}
    \item 推行Rust
        \begin{itemize}
            \item Rust的性能和C/C++齐平的同时强调内存安全。
            \item 这里是我写的一个面向架构师的介绍:\url{https://github.com/TimeExceed/cert_rust}
        \end{itemize}
    \item 在有条件的情况下使用容器技术,方便部署和升级。
    \item 核心逻辑和展示层分离,供应商提供的逻辑和自己的逻辑分离。
        分离到不同的容器里,至少是不同的进程里。
        相互之间以unix domain socket通讯。
        其背后的思想是:一,面对不同的进化压力的组件不宜合在一个进程内;二,责任切分清楚,谁的代码出稳定性问题一目了然。
\end{itemize}
管理手段包括:
\begin{itemize}
    \item 干掉软件测试团队,开发自己测试自己发布。
    \item 强调回归测试。
    \item 强调设计review和代码review。
    \item 强调“未知死,焉知生”,意思是,如果开发没有想过自己的代码什么时候会死,死了之后谁来处理、如何处理,那么这份代码也不会工作正常。
\end{itemize}
以上这些措施背后的理论思考我总结在\href{https://github.com/TimeExceed/design-principles/blob/master/se-principles.notes.pdf}{这份幻灯片}里了。
最终的成果是,
\begin{itemize}
    \item 依图的硬件产品没有一次内存问题导致的core dump发生在我团队的代码里。
    \item 我团队代码对性能的overhead几乎看不到。
        比如在海思3559上依图算法做到16路人脸抓拍,包上我的业务代码之后,还是16路人脸抓拍。
\end{itemize}

\subsection{云上关注接入设备的数量和多样性}

云上的关注点有两个:一是接入的设备绝对数量多,二是接入设备的类型多。
\begin{itemize}
    \item 设备的类型多意味着,设备接入应该是云上独立的一层。
        这一层对云上业务暴露统一的接口,屏蔽掉设备类型不同导致的协议差异。
        包括由设备主动发起连接还是由云主动发起连接的差异。
    \item 接入设备的绝对数量多,我的应对方法是将接入层分成调度节点和接入节点。
        如果云被动接受设备的连接,那么还有分发节点。
        接入节点本身无状态,完全根据调度节点的指令工作,因而可以非常轻量地水平扩展。
        而通常接入设备的数量的变化并不大,并且设备之间数据流量的差异和同一设备的数据流量在时间上也较平稳,因此调度节点做调度决策可以慢慢来。
        于是调度节点的开发难度降低不少。
\end{itemize}

\subsection{云和端之间的网络的可达性和性能}

AIoT和传统传感器IoT场景相同的点在于,数据以流式数据为主,间杂以少量控制指令。
而不同的点在于
\begin{itemize}
    \item 单一AIoT设备的数据量更大、数据流更多。
    \item AIoT的网络环境通常更不可控。
        有时网络稳定性很差。
\end{itemize}
针对这些问题,我在依图的实践采用HTTP/2+TLS。
\begin{itemize}
    \item HTTP/2的长连接特性可以减少TCP冷启动带来的吞吐量下降。
    \item HTTP/2的单连接多流特性也满足AIoT设备需求,不必在应用层硬将多条数据流编码进一个数据流。
    \item 同时HTTP/2的其他细节比如ping frame和流控也和AIoT的场景相匹配。
\end{itemize}
并且还留有一个HTTP/3(即HTTP/2 over UDP)以应对更极端的弱网环境的后手。

\section{履历}

\subsection*{2020.5-2021.5 依图/基础设施}

\begin{itemize}
    \item 三人领导小组成员。
        分管软件研发。
        推动部门向内部服务商转型,要点是谈钱。
        \begin{itemize}
            \item Philosophy, principle, plan三人各取一端。
                我是那个philosopher。
        \end{itemize}
    \item 期间领导了依图自用的PaaS(K8S)、训练用存储系统的研发。
\end{itemize}

\subsection*{2018.6-2020.4 依图/硬件产品}

\begin{itemize}
    \item 主持多款硬件产品(门禁、摄像头、边缘盒子)的软件研发。
        \begin{itemize}
            \item 力主转换技术栈到Rust。
                达到了no crash和low overhead两个目标。
        \end{itemize}
    \item 设计并写作依图统一API(OneAPI)
\end{itemize}

\subsection*{2013.12-2018.5 阿里云/表格存储}

\begin{itemize}
    \item 期间将Software Transactional Memory和Immutable Data Structures引入表格存储。
    \item 作为架构师参与女娲向全球一致性(内部)服务转型。
\end{itemize}

\subsection*{2013.5-2013.11 费沙科技}

\begin{itemize}
    \item 作为技术负责人参与别人创业。后创业失败。
    \item 期间选择Clojure和ClojureScript作为后端和前端的核心技术栈。
    \item 在Docker的早期就开始关注,并用于CI。
\end{itemize}

\subsection*{2011.9-2013.4 阿里云/飞天/伏羲-女娲}

\begin{itemize}
    \item P7入职,半年后升P8。
        至今我的晋升PPT还作为升职范本在阿里内部流传。
    \item 作为架构师参与伏羲(分布式调度)重构,为后面支持5K集群做好铺垫。
    \item 主管女娲(一致性)。
        主持女娲迁移Zookeeper。
\end{itemize}

\subsection*{2007-2011.8 纳拓科技}

\begin{itemize}
    \item 一家EDA公司,后被Cadence收购。我以实习生身份加入。日常使用C++开发。
    \item 作了大量的talks。
\end{itemize}

\subsection*{1998-2011 上海交通大学}

\begin{itemize}
    \item 2000年,作为参赛队员参加ACM/ICPC World Final,获得第7名。实现亚洲队伍前十的突破。
    \item 2002年,交大ACM队获得总决赛冠军。我是常务教练。
    \item 博士期间师从沈恩绍教授,学习数理逻辑和自动机理论。
        中后期转到博弈论、拍卖论和机制设计方向,并以在线广告相关研究获得学位。
\end{itemize}

\appendix
\clearpage
\section{专利}

\subsection*{依图时期}

\begin{enumerate}
    \item CN111770491A/CN202010511943.6 一种数据链路建立方法及装置
    \item CN111581410A/CN202010475291.5 图像检索方法及其装置、介质和系统
    \item CN111615070A/CN202010467753.9 一种业务处理方法及装置
    \item CN111563926A/CN202010446016.0 测量图像中物体物理尺寸的方法、电子设备、介质及系统
    \item CN111586066A/CN202010397155.9 一种多媒体数据加密处理的方法及装置
    \item CN111414497A/CN202010348078.8 一种分布式数据融合处理的方法及装置
    \item CN111510658A/CN202010326765.X 图片数据处理方法、装置、电子设备及存储介质
    \item CN111491369A/CN202010317690.9 一种基于安防前端设备的终端定位方法及装置
    \item CN111399886A/CN202010285926.5 用于设备快速升级的方法及系统
    \item CN111447463A/CN202010254256.0 视频流格式转换及播放控制的方法、装置、介质及其系统
    \item CN111444373A/CN202010234745.X 图像检索方法及其装置、介质和系统
    \item CN110662047A/CN201910956997.0 图像存储方法及装置、电子设备以及计算机存储介质
    \item CN110677699A/CN201910956998.5 视频流和/或图片流数据的共享方法、装置及电子设备
    \item CN110582010A/CN201910939027.X 视频/图片加密传输方法、装置、电子设备及存储介质
    \item CN110717428A/CN201910923395.5 一种融合多个特征的身份识别方法、装置、系统、介质及设备
    \item CN110730216A/CN201910904367.9 图片推送方法、装置、电子设备及计算机可读存储介质
\end{enumerate}

\subsection*{阿里时期}

\begin{enumerate}
    \item CN110019514A/CN201711083268.6 数据同步方法、装置以及电子设备
    \item CN107438092A/CN201710140500.9 用于分布式场景中数据处理的方法和设备
    \item CN107438092B/CN201710140500.9 用于分布式场景中数据处理的方法和设备
    \item CN107196988A/CN201710139662.0 一种跨地域数据传输的方法和设备
    \item CN107196988B/CN201710139662.0 一种跨地域数据传输的方法和设备
    \item CN107145396A/CN201710117678.1 分布式锁实现方法和设备
    \item CN107145396B/CN201710117678.1 分布式锁实现方法和设备
    \item CN108345617A/CN201710060057.4 一种数据同步方法、装置以及电子设备
    \item CN108347455A/CN201710053030.2 元数据交互方法及系统
    \item CN108347454A/CN201710053029.X 元数据交互方法及系统
    \item CN108347454B/CN201710053029.X 元数据交互方法及系统
    \item CN107438092B/CN:201710140500:A Data processing method and apparatus applied to distributed scenes
    \item CN107196988B/CN:201710139662:A Cross-area data transmission method and equipment
    \item CN107145396B/CN:201710117678:A Distributed lock realization method and equipment
    \item CN108347455B/CN201710053030.2 元数据交互方法及系统
    \item CN108280080A/CN201710009380.9 一种数据同步方法、装置以及电子设备
    \item CN107919977A/CN201610888948.4 一种基于Paxos协议的分布式一致性系统的在线扩容、在线缩容的方法和装置
    \item CN107872395A/CN201610843856.4 流量控制方法及设备
    \item CN107786527A/CN201610792479.6 实现服务发现的方法及设备
    \item CN107783860A/CN201610792603.9 一种数据传输的恢复点目标监控方法及设备
    \item CN107800733A/CN201610791965.6 分布式系统中会话标识的生成方法及设备
    \item CN107800733B/CN201610791965.6 分布式系统中会话标识的生成方法及设备
    \item CN107800733B/CN:201610791965:A Generating method of session identification in distributed system and equipment XXX
    \item CN107517227A/CN201610424538.4 用于分布式一致性系统的会话实现方法以及装置
    \item CN107517227B/CN201610424538.4 用于分布式一致性系统的会话实现方法以及装置
    \item CN107124324A/CN201610105054.3 一种基于租约的心跳协议方法和设备
    \item CN107124324B/CN201610105054.3 一种基于租约的心跳协议方法和设备
    \item CN107517227B/CN:201610424538:A Session implementation method and device for distributed consistency system
    \item CN107124324B/CN:201610105054:A Heartbeat protocol method and device based on lease
    \item CN106933547A/CN201511017566.6 全局信息获取及处理的方法、装置和更新系统
    \item CN106933548A/CN201511017614.1 全局信息获取、处理及更新、方法、装置和系统
    \item CN106933550A/CN201511017784.X 全局信息获取、处理及更新方法、装置和系统
    \item CN106933547B/CN201511017566.6 全局信息获取及处理的方法、装置和更新系统
    \item CN106933550B/CN201511017784.X 全局信息获取、处理及更新方法、装置和系统
    \item CN106933548B/CN201511017614.1 全局信息获取、处理及更新、方法、装置和系统
    \item CN106933547B/CN:201511017566:A Global information obtaining method and apparatus, global information processing method and apparatus, and global information updating system
    \item CN106933550B/CN:201511017784:A Method and device for obtaining, processing and updating global information and global information update system
    \item CN106933548B/CN:201511017614:A Global information obtaining method, apparatus and system, global information processing method, apparatus and system, and global informatin updating method, apparatus and system
    \item CN106572130A/CN201510648700.6 用于实现分布式锁管理的方法和设备
    \item CN106572051A/CN201510648902.0 分布式系统中分布式锁服务实现方法以及装置
    \item CN106572054A/CN201510650637.X 分布式系统中分布式锁服务实现方法以及装置
    \item CN106572054B/CN201510650637.X 分布式系统中分布式锁服务实现方法以及装置
    \item CN106572130B/CN201510648700.6 用于实现分布式锁管理的方法和设备
    \item CN106572054B/CN:201510650637:A Distributed lock service realization method and device for distributed system
    \item CN106572130B/CN:201510648700:A Method used for realizing distributed lock management and equipment thereof
    \item CN106534227A/CN201510570429.9 用于扩展分布式一致性服务的方法和设备
    \item CN106534227B/CN201510570429.9 用于扩展分布式一致性服务的方法和设备
    \item CN106453444A/CN201510476786.9 缓存数据共享的方法及设备
    \item CN106453444B/CN201510476786.9 缓存数据共享的方法及设备
    \item CN106406925A/CN201510481261.4 用于支持在线升级的设备和方法
    \item CN106534227B/CN:201510570429:A Method and device of expanding distributed consistency service
    \item CN106453444B/CN:201510476786:A Cache data sharing method and equipment
\end{enumerate}

\end{document}
